\section{Ketidakjujuran Akademik}
\label{sec:ketidakjujuranakademik}

Berkebalikan dengan integritas akademik, ketidakjujuran akademik didefinisikan sebagai perilaku tidak etis dalam lingkungan akademik \citep{muhammad2020factors}. Ini adalah perilaku yang tidak pantas dimana siswa bertindak untuk mendapatkan keuntungan akademik yang tidak adil untuk diri mereka sendiri atau untuk teman-teman mereka di komunitas akademik \citep{grira2019rationality}. Ketidakjujuran akademik mencegah perkembangan nilai-nilai positif seperti kejujuran, keadilan dan kemajuan belajar yang signifikan, dan terkait dengan perilaku negatif lainnya, yang memiliki implikasi bahkan di luar akademik \citet{krou2020achievement} \citep{yu2018college}, seperti di pasar kerja di mana lulusan dengan keterampilan yang tidak tepat dapat dipekerjakan \citep{barbaranelli2018machiavellian} \citep{bashir2018development}.

Penelitian menunjukkan bahwa perilaku seperti itu adalah fenomena yang diketahui dan lazim yang telah meningkat selama beberapa tahun terakhir \citep{birks2020managing} \citep{grira2019rationality} \citep{harper2021detecting}, dan juga bahwa ini adalah fenomena global lintas budaya yang memiliki banyak segi. Misalnya, penelitian di India menemukan bahwa sedikit lebih dari 20\% dari 1.369 peserta penelitian mengakui ketidakjujuran akademis \citep{stearns2001student}. Demikian pula, salah satu studi terluas dan terlama yang dilakukan di Australia memeriksa 150.000 siswa selama delapan tahun dan menemukan bahwa 65\% siswa melaporkan ketidakjujuran akademik dalam setidaknya satu parameter penelitian \citep{duff2006international}. Demikian pula, penelitian yang dilakukan di Rumania menemukan bahwa 95\% siswa melaporkan perilaku akademik yang tidak pantas \citep{ives2017patterns}.

Pada literatur-literatur penelitian digambarkan sejumlah besar perilaku yang berhubungan dengan perilaku akademik yang tidak pantas dalam lingkungan pembelajaran tradisional non-online, termasuk: membantu teman dalam ujian, bekerja sama dengan teman sebaya selama ujian, penggunaan larangan bahan ujian, penggunaan bahan teman, mengizinkan pekerjaan untuk disalin, mendapatkan solusi dari teman yang telah mengikuti ujian, mengikuti ujian untuk orang lain, plagiarisme (termasuk materi yang disalin tanpa memberikan kredit kepada penulis, penggunaan berulang tugas yang sudah diserahkan, karya yang ditulis oleh pihak ketiga dan disajikan sebagai karya siswa atau membeli karya – menyontek kontrak), kerjasama antar teman untuk menulis karya ketika tidak ada izin untuk melakukannya dan menambahkan sumber ke daftar pustaka tanpa menggunakannya \citep{denisova2017challenges} \citep{harper2021detecting} \citep{von2001can} \citep{yu2018college}.  

Sebuah studi baru-baru ini melaporkan bahwa sebagian besar perilaku yang dianggap kurang integritas akademik terkait dengan bantuan selama ujian, yang paling umum adalah memberi dan menerima bantuan teman dalam ujian pilihan ganda dan dalam ujian di mana jawaban singkat diperlukan \citep{harper2021detecting}.

Selain itu, terdapat suatu peningkatan motivasi siswa untuk membayar faktor luar untuk mengerjakan tugas (kecurangan kontrak) \citep{birks2020managing}. Ditemukan bahwa alasan utama penggunaan kecurangan kontrak adalah ketidakpuasan dengan lingkungan belajar mengajar, kurangnya waktu dan persepsi bahwa ada banyak peluang untuk menyontek \citep{bretag2019contract} \citep{foltynek2018analysis}. Di Australia, ditemukan bahwa siswa menggunakan kecurangan kontrak yang disediakan oleh lingkaran sosial terdekat mereka daripada kontraktor eksternal \citep{harper2021detecting}.

Sebuah meta-analisis yang baru-baru ini diterbitkan oleh \citet{krou2020achievement} dari berbagai studi penelitian yang menyelidiki antara lain, perilaku di berbagai bidang (sains, teknologi, teknik, matematika, dan aliran bisnis) mengkategorikan perilaku terkait ketidakjujuran akademik menjadi dua kategori: plagiarisme (materi disalin tanpa memberikan kredit kepada penulis, karya yang ditulis oleh pihak ketiga dan dipresentasikan sebagai hasil karya siswa dan lain-lain) dan mencontek (menerima jawaban dari siswa yang sudah menyelesaikan ujian, mengerjakan tugas bersama teman tanpa izin, menyalin saat ujian, dan menggunakan bahan bantu tanpa izin selama ujian sedang berlangsung). Selain itu, ditemukan bahwa siswa yang menyaksikan perilaku akademik yang tidak pantas dari temannya cenderung melakukan perilaku serupa, berbeda dengan siswa yang tidak menyaksikan perilaku tersebut \citep{ahmed2018student} \citep{barbaranelli2018machiavellian} \citep{kiekkas2020reasons}. Dengan kata lain, norma perilaku akademik yang pantas atau tidak pantas mempengaruhi perilaku siswa. Penyebab ketidakjujuran akademik siswa banyak dan beragam dan bersumber dari faktor pribadi-intrinsik atau ekstrinsik. Faktor intrinsik pribadi meliputi motivasi yang kuat untuk berhasil, daya saing, takut gagal, pengetahuan yang tidak memadai dalam disiplin, rasa efikasi diri yang berkurang, studi yang berlebihan, disiplin diri yang tidak memadai, kemalasan, kelelahan, kecenderungan impulsif, sebelumnya. prestasi akademik yang rendah dan perkembangan moral yang rendah. Faktor ekstrinsik termasuk pengabaian perilaku tidak etis oleh anggota staf dan tidak adanya implikasi disiplin untuk menyontek, tekanan orang tua untuk berhasil, ketidakpuasan dengan pengajaran, perasaan bahwa ada banyak peluang menyontek, tekanan waktu untuk menyerahkan tugas, akademik yang terlalu tinggi. tuntutan, konten yang tidak relevan dengan profesi masa depan siswa, keinginan untuk mencapai status sosial yang lebih baik dan keinginan untuk memasuki pasar kerja \citep{amigud2019246} \citep{birks2020managing} \citep{bretag2019contract} \citep{kiekkas2020reasons} \citep{krou2020achievement} \citep{murdock2006motivational}.

Alasan perilaku ini dapat dikategorikan menurut tiga mekanisme motivasi: (1) Apa tujuan saya? Ini termasuk pertimbangan motivasi intrinsik dan ekstrinsik siswa; (2) Bisakah saya melakukan ini? Ini termasuk motivasi ekstrinsik siswa, efikasi diri dan lingkungan belajar mereka, termasuk ketidakmampuan belajar, ujian yang tidak jelas, dan keinginan untuk menjadi seperti peserta didik lainnya \citep{bertram2017academic} \citep{etgar2019white} \citep{murdock2006motivational}; dan (3) Berapa biayanya? Ini termasuk pertimbangan biaya langsung dari tertangkap tetapi juga beban psikologis dari perilaku akademik yang tidak jujur \citep{bertram2017academic} \citep{etgar2019white} \citep{murdock2006motivational}.

Ketidakjujuran akademik ditemukan berkorelasi positif dengan motivasi ekstrinsik \citep{barbaranelli2018machiavellian} \citep{grira2019rationality} \citep{krou2020achievement} \citep{murdock2006motivational} dan berkorelasi negatif dengan motivasi intrinsik \citep{foltynek2018analysis} \citep{barbaranelli2018machiavellian} \citep{grira2019rationality} \citep{murdock2006motivational}. Akan tetapi, bukan hanya motivasi yang mempengaruhi perilaku tersebut.

Literatur di bidang ini menunjukkan bahwa persepsi siswa tentang ketidakjujuran akademik dapat menjelaskan beberapa perilaku tersebut \citep{kiekkas2020reasons}. Pernyataan seperti "itu bukan masalah besar", "ini tidak benar-benar menyontek", "ini salah guru saya", atau "semua orang curang" \citep{stephens2007does} adalah contoh persepsi siswa tentang kurangnya integritas sebagai sesuatu yang tidak serius. Selain itu, siswa tidak selalu mempersepsikan perilaku tertentu, seperti penggunaan bahan tanpa memperhatikan sumbernya \citep{moss2018systematic} atau penggunaan catatan tersembunyi dalam ujian, sebagai karakteristik perilaku ketidakjujuran akademik \citep{kiekkas2020reasons}.

Berbeda dengan sikap siswa, staf pengajar menganggap ketidakjujuran akademik jauh lebih serius \citep{blau2021violation} \citep{stevens2013promoting}. \citet*{pincus2003faculty} bahkan menemukan bahwa dosen menganggap perilaku seperti menyalin dalam ujian, penggunaan materi terlarang selama ujian, mengikuti ujian untuk orang lain dan membayar seseorang untuk menulis makalah sebagai perilaku yang sangat tidak pantas. Kegagalan untuk berkontribusi pada kerja kelompok, berbohong dan menyajikan makalah yang sama di lebih dari satu mata kuliah adalah semua perilaku yang dianggap kurang parah oleh dosen. Namun, secara umum, dosen menganggap semua perilaku yang menunjukkan ketidakjujuran akademik lebih serius daripada mahasiswa, dan menganggap lebih banyak perilaku sebagai manifestasi ketidakjujuran akademik daripada mahasiswa.

Ketidakjujuran akademik telah menyibukkan akademisi selama bertahun-tahun, tetapi fenomena ini telah meningkat dalam beberapa tahun terakhir. Salah satu alasan peningkatan ini adalah pertumbuhan pengajaran online, dan teknologi yang memfasilitasi perilaku ini \citep{etgar2019white} \citep{peytcheva2018impact} \citep{sarwar2018paid}.

Dalam dekade terakhir, pendekatan pembelajaran yang inovatif telah diperkenalkan ke dalam sistem pendidikan tinggi. Perkembangan teknologi dan penggunaannya yang umum telah mendorong institusi pendidikan tinggi untuk memperkenalkan kursus online, baik kursus online atau hybrid, ke dalam program pembelajaran akademik mereka \citep{lee2017online} \citep{marshall2017attack}. Pendekatan ini memungkinkan peningkatan akses penuh dan mudah ke konten pembelajaran, penggunaan media sosial, Wikipedia, situs berbagi, dll \citep{ahmed2018student} \citep{lee2017online} \citep{peytcheva2018impact}. Bahkan, teknologi digital seperti Smartphone, komputer palm, komputer mobile dan PC dan Internet memungkinkan lebih banyak fleksibilitas, kreativitas dan kadang-kadang bahkan akurasi dan efektivitas. Oleh karena itu, mereka membantu proses belajar-mengajar, karena memungkinkan fotografi dan penyimpanan berbagai materi pembelajaran \citep{peytcheva2018impact} \citep{stephens2007does}, berbagi pengetahuan dan integrasi berbagai metode untuk membuat pembelajaran lebih aktif dan terlibat. 

Namun, integrasi kursus online tanpa mengintegrasikan aturan untuk perilaku etis yang sesuai untuk lingkungan online dan teknik khusus untuk mencegah ketidakjujuran akademik memberikan lahan subur bagi peningkatan frekuensi perilaku akademik yang tidak pantas \citep{marshall2017attack}.Selain itu, keuntungan mengintegrasikan teknologi dalam pembelajaran (kenyamanan, fleksibilitas dan akses ke informasi) menjadi insentif terbesar untuk perilaku tidak jujur \citep{blau2017ethical} \citep{muhammad2020factors} \citep{peytcheva2018impact}. Contohnya adalah plagiarisme, yang—karena akses informasi yang mudah—menjadi mudah digunakan dengan copy-paste, jauh lebih mudah daripada menyalin \citep{sidi2019ethical}.

Penelitian menunjukkan bahwa siswa mulai menggunakan alat teknologi yang tidak sah seperti Smartwatch dan Smartphone untuk perilaku ini \citep{birks2020managing} \citep{blau2017ethical}. Kemudian juga ditemukan bahwa dosen dan mahasiswa sama-sama percaya bahwa menyontek lebih mudah di kursus online \citep{kennedy2000academic}.

Dalam lingkungan belajar tanpa pengawasan, ada penjelasan tambahan untuk perilaku tidak etis setelah integrasi teknologi \citep{peytcheva2018impact}, seperti kelebihan aplikasi berbasis internet yang dapat diakses oleh mahasiswa, akses mudah ke dukungan tidak sah dari luar kampus (outsourcing), interaksi tatap muka yang tidak memadai dengan staf pengajar dalam kursus online yang mengarah pada penurunan komitmen moral, umpan balik yang tidak memadai tentang kegiatan belajar akademik, pedoman yang tidak tepat bagi siswa dalam perjalanan belajar online, kurangnya pelatihan yang sesuai untuk pembelajaran online dan kurangnya mekanisme pemantauan yang tepat \citep{von2001can}. Namun demikian, perlu dicatat bahwa teknologi itu sendiri bukanlah penyebab perilaku tidak jujur, itu hanya mempermudah dan memungkinkannya terjadi \citep{blau2017ethical} \citep{etgar2019white} \citep{sarwar2018paid}. Ada juga beberapa teknik yang dirancang khusus yang efektif untuk mencegah ketidakjujuran akademik, faktor yang juga dapat menjelaskan peningkatan prevalensi perilaku akademik yang tidak pantas \citep{marshall2017attack}.

Penelitian juga menunjukkan bahwa banyak perilaku yang dianggap sebagai ketidakjujuran akademik dan terkait dengan perangkat digital berasal dari kurangnya pengetahuan dan pemahaman siswa tentang perilaku etis \citep{blau2017ethical}. Misalnya, "copy-paste" tidak selalu dianggap sebagai praktik yang tidak etis. Poin penting lainnya adalah persepsi akademik tentang ketidakjujuran akademik di ruang digital kurang berbahaya daripada ketidakjujuran akademik di ruang akademik analog, karena di ruang digital dianggap sebagai "kejahatan kerah putih" \citep{etgar2019white} dan karena itu dianggap kurang merusak. Dengan demikian, hukuman untuk perilaku ini lebih ringan dibandingkan dengan hukuman di lembaga akademis yang serupa \citep{etgar2019white}.