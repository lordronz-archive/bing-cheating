\section{Ketidakjujuran Akademik}
\label{sec:ketidakjujuranakademik}

Berkebalikan dengan integritas akademik, ketidakjujuran akademik didefinisikan sebagai perilaku tidak etis dalam lingkungan akademik \citep{muhammad2020factors}. Ini adalah perilaku yang tidak pantas dimana siswa bertindak untuk mendapatkan keuntungan akademik yang tidak adil untuk diri mereka sendiri atau untuk teman-teman mereka di komunitas akademik \citep{grira2019rationality}. Ketidakjujuran akademik mencegah perkembangan nilai-nilai positif seperti kejujuran, keadilan dan kemajuan belajar yang signifikan, dan terkait dengan perilaku negatif lainnya, yang memiliki implikasi bahkan di luar akademik \citet{krou2020achievement} \citep{yu2018college}, seperti di pasar kerja di mana lulusan dengan keterampilan yang tidak tepat dapat dipekerjakan \citep{barbaranelli2018machiavellian} \citep{bashir2018development}.

Penelitian menunjukkan bahwa perilaku seperti itu adalah fenomena yang diketahui dan lazim yang telah meningkat selama beberapa tahun terakhir \citep{birks2020managing} \citep{grira2019rationality} \citep{harper2021detecting}, dan juga bahwa ini adalah fenomena global lintas budaya yang memiliki banyak segi. Misalnya, penelitian di India menemukan bahwa sedikit lebih dari 20\% dari 1.369 peserta penelitian mengakui ketidakjujuran akademis \citep{stearns2001student}. Demikian pula, salah satu studi terluas dan terlama yang dilakukan di Australia memeriksa 150.000 siswa selama delapan tahun dan menemukan bahwa 65\% siswa melaporkan ketidakjujuran akademik dalam setidaknya satu parameter penelitian \citep{duff2006international}. Demikian pula, penelitian yang dilakukan di Rumania menemukan bahwa 95\% siswa melaporkan perilaku akademik yang tidak pantas \citep{ives2017patterns}.

Pada literatur-literatur penelitian digambarkan sejumlah besar perilaku yang berhubungan dengan perilaku akademik yang tidak pantas dalam lingkungan pembelajaran tradisional non-online, termasuk: membantu teman dalam ujian, bekerja sama dengan teman sebaya selama ujian, penggunaan larangan bahan ujian, penggunaan bahan teman, mengizinkan pekerjaan untuk disalin, mendapatkan solusi dari teman yang telah mengikuti ujian, mengikuti ujian untuk orang lain, plagiarisme (termasuk materi yang disalin tanpa memberikan kredit kepada penulis, penggunaan berulang tugas yang sudah diserahkan, karya yang ditulis oleh pihak ketiga dan disajikan sebagai karya siswa atau membeli karya – menyontek kontrak), kerjasama antar teman untuk menulis karya ketika tidak ada izin untuk melakukannya dan menambahkan sumber ke daftar pustaka tanpa menggunakannya \citep{denisova2017challenges} \citep{harper2021detecting} \citep{von2001can} \citep{yu2018college}.  

Sebuah studi baru-baru ini melaporkan bahwa sebagian besar perilaku yang dianggap kurang integritas akademik terkait dengan bantuan selama ujian, yang paling umum adalah memberi dan menerima bantuan teman dalam ujian pilihan ganda dan dalam ujian di mana jawaban singkat diperlukan \citep{harper2021detecting}.
