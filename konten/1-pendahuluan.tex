% Ubah judul dan label berikut sesuai dengan yang diinginkan.
\section{Pendahuluan}
\label{sec:pendahuluan}

% Ubah paragraf-paragraf pada bagian ini sesuai dengan yang diinginkan.

Pandemi COVID-19 telah melihat gerakan internasional menuju pengajaran dan penilaian online. Perpindahan secara online sering kali diselesaikan dalam waktu singkat dan dengan sedikit peluang untuk membuat rencana guna memastikan integritas akademik tetap terjaga. Banyak metode penilaian standar yang digunakan untuk memungkinkan instruktur mengevaluasi kompetensi, keterampilan dan pengetahuan siswa, termasuk mengadakan ujian dan mengharuskan siswa untuk menulis makalah \citet{stiggins2017perfect} \citet{wiggins2011true}. Hasil yang diperoleh siswa selama penilaian ini dapat menjadi sangat penting untuk kehidupan dan karir masa depan mereka, menentukan status ekonomi dan posisi mereka dalam masyarakat. \citet{fontaine2020exam}.

Pandemi Covid-19 memasuki kehidupan kita dengan sangat tiba-tiba, menyebar luas dan cepat selama tahun 2020. Hal itu memaksa sekolah, universitas, dan lembaga pendidikan lainnya tutup selama beberapa bulan. Untuk memungkinkan kelanjutan kegiatan belajar-mengajar selama periode yang sulit ini, para pendidik di seluruh dunia pindah dari pengajaran dan pembelajaran di kelas ke kelas darurat yang jauh atau pembelajaran jarak jauh \citet{rebecca2020}. Meskipun beberapa sistem pendidikan telah kembali ke pengajaran frontal, sistem pendidikan tinggi di sebagian besar negara, termasuk Israel, berlanjut dan berlanjut pada saat artikel ini ditulis untuk melakukan pembelajaran jarak jauh, dan oleh karena itu pembelajaran online menjadi satu-satunya kemungkinan untuk memastikan kelanjutan dari pengajaran dan pembelajaran akademik, secara luas dan tanpa batasan waktu dan tempat. Di akhir semester pertama pembelajaran jarak jauh, ujian dan evaluasi siswa juga diubah menjadi metode online.

Sifat daripada penilaian yang sedemikian rupa sehingga, apakah itu diadakan secara langsung atau online, siswa memiliki insentif pribadi untuk mencoba dan mendapatkan nilai terbaik yang mereka bisa. Ini berarti bahwa beberapa orang mungkin menggunakan cara yang tidak adil, atau seperti yang dinyatakan oleh literatur integritas akademik, mereka dapat bertindak dengan ketidakjujuran akademik atau melakukan pelanggaran akademik. Dalam situasi lain, pelanggaran integritas tersebut dapat dicap sebagai kecurangan atau penipuan.

Salah satu masalah evaluasi online adalah bahwa berbagai kemungkinan teknologi memfasilitasi perilaku non-etis, seperti berbagi informasi di Internet, berkonsultasi dengan teman, dan menyalin konten dengan mudah \citet{peytcheva2018impact} \citet{sarwar2018paid}. Memang, literatur penelitian membahas ketidakjujuran akademik siswa dalam pembelajaran online, termasuk menyalin, melarang penggunaan materi pembelajaran, membantu orang lain, dll \citet{ahmed2018student} \citet{birks2020managing} \citet{grira2019rationality} \citet{stearns2001student}. Literatur juga membahas persepsi dosen tentang ketidakjujuran akademik mahasiswa \citet{blau2021violation} \citet{pincus2003faculty} \citet{stevens2013promoting}, meskipun pada tingkat yang lebih rendah. Namun telah ditemukan bahwa dosen pada umumnya mempersepsikan ketidakjujuran akademik lebih parah dibandingkan dengan persepsi mahasiswa \citep{blau2021violation} \citep{pincus2003faculty}. Namun, baik mahasiswa maupun dosen percaya bahwa menyontek lebih mudah di kursus online \citet{kennedy2000academic}. Hal ini merupakan fenomena yang mengkhawatirkan karena menyontek memiliki konsekuensi baik bagi proses belajar siswa selama studi akademis mereka maupun untuk pasar kerja yang akan mereka terima setelah lulus, dengan etika yang mereka bawa ke pasar itu \citep{barbaranelli2018machiavellian} \citep{bashir2018development}. Ujian dan tugas sekarang semuanya dilakukan secara online atau daring dan tampaknya ujian dan tugas online ini akan dilakukan setidaknya di masa mendatang. Oleh karena itu, menjadi keharusan untuk menilai fenomena ini, terutama karena pembelajaran semakin banyak dilakukan secara online dan terutama selama pandemi Covid-19.

Ketidakjujuran akademik bukanlah fenomena baru. Itu dimulai jauh sebelum teknologi memasuki kehidupan kita dan mencakup berbagai perilaku. Penelitian ini bertujuan untuk mengetahui perbedaan sikap mahasiswa dan dosen terhadap ujian online selama krisis Covid-19, alasan (motivasi) mahasiswa melakukan ketidakjujuran akademik selama periode ini, kesaksian mahasiswa tentang perilaku aktual mereka selama ujian online dan apakah Ada korelasi antara sikap siswa terhadap ujian online dan alasan ketidakjujuran akademik mereka dalam ujian ini selama periode Covid-19.

Ketidakjujuran dalam praktik akademik, khususnya menyontek saat ujian, tersebar luas di perguruan tinggi dan kampus universitas di seluruh dunia \citep{chen2009relationship}. Karena itu, Universitas Bangladesh tidak terkecuali untuk fenomena ketidakjujuran akademik menyontek ujian online. Kebutuhan akan nilai tinggi semakin mendominasi budaya Bangladesh pendidikan, bersama dengan tekanan untuk menyelesaikan a gelar dengan pujian dan untuk mendapatkan pekerjaan di elit organisasi \citep{razek2014academic}. Selain tekanan pada siswa yang disebabkan oleh banyak ini harapan, karantina COVID-19, yang telah menciptakan ketidakpastian bagi manusia di seluruh dunia, telah menambah stres dan kecemasan pada siswa dan memunculkan kondisi anomik yang mengakibatkan siswa menjadi egois berkaitan dengan pertimbangan etis \citep{memon2020analysis}. Pada penelitian ini bertujuan untuk mengidentifikasi alasan di balik maraknya kecurangan ujian online selama Pandemi COVID-19, menjelaskan bagaimana kenaikannya kecurangan ujian online terjadi, dan berkembang kerangka teoritis untuk kuantitatif lebih lanjut studi penelitian.

Beberapa penelitian telah dilakukan pada pengaruh pandemi COVID-19 terhadap tingkat kecemasan, stres, dan mahasiswa universitas depresi ketika mereka harus mengambil kelas mereka on line. Misalnya, pembelajaran online, yang memiliki telah diimplementasikan secara global di universitas dan perguruan tinggi sebagai alternatif pendidikan di kampus, telah menyebabkan stres, depresi, dan kecemasan, dari ringan hingga sedang hingga berat, di kalangan universitas siswa \citep{fawaz2021learning} ,\citep{kecojevic2020impact}. Sebuah studi baru-baru ini diterbitkan oleh AlAteeq, Aljhani, dan AlEesa \citep{alateeq2020perceived} tentang stres tingkat mahasiswa di Arab Saudi selama penguncian COVID-19 telah menemukan korelasi yang signifikan dan positif antara tinggi tingkat stres dan mahasiswi di dunia maya ruang kelas. Juga, menurut sebuah studi baru-baru ini oleh Islam, Barna, Raihan, Khan, dan Hossain \citep{islam2020depression}, mahasiswa telah menderita penyakit sedang dan stres dan kecemasan yang parah karena penguncian COVID-19, dan penelitian lain menemukan bahwa pandemi telah menyebabkan masalah kesehatan mental, seperti stres, depresi, dan kecemasan \citep{aylie2020psychological}. Sosial isolasi, jarak sosial, dan kejatuhan ekonomi juga dapat memicu frustrasi, gugup, rasa bersalah, marah, khawatir, sedih, takut, jengkel, kesepian, dan perasaan tidak berdaya, dan ini juga telah diidentifikasi sebagai faktor yang mempengaruhi kecurangan di kalangan mahasiswa. Ini penelitian sebelumnya telah menemukan korelasi antara COVID-19 dan stres, kecemasan, dan depresi atau mampu menghubungkan stres, depresi, dan kecemasan untuk selingkuh. Namun, penelitian sebelumnya ini tidak mengidentifikasi COVID-19 adalah faktor eksternal yang mempengaruhi e-cheating dan jelaskan bagaimana bentuk kecurangan khusus ini bisa terjadi sebagai akibat dari stres dan kecemasan. Selain itu, penelitian sebelumnya belum mengembangkan kerangka kerja apa pun yang melaluinya peneliti dapat memahami hubungan antara pandemi COVID-19 dan online kecurangan ujian antara universitas dan perguruan tinggi siswa. Studi penelitian ini mencoba untuk mengisi kesenjangan dalam literatur dan mengembangkan teori kerangka kerja yang menjelaskan hubungan antara COVID-19, stres, kecemasan, dan e-cheating. Itu fokus penelitian ini adalah pada pengaruh bahwa karantina selama pandemi COVID-19 telah pernah menyontek siswa pada ujian online dan beberapa bentuk lain dari menyontek pada ujian digital, termasuk mengakses situs web terkait ujian, menggunakan ponsel dan platform media sosial selama ujian, menyimpan jawaban ujian di komputer mereka, dan memiliki Bus Serial Universal non-ujian (USB) dengan port yang berisi jawaban dan solusi untuk soal ujian \citep{baker2008student}.

Berfokus pada karantina COVID-19 sebagai penyebab stres dan kecemasan, penulis ini studi penelitian membahas tiga hal berikut: pertanyaan. Memiliki karantina COVID-19 mempengaruhi kecurangan ujian online, khususnya menyontek, di kalangan mahasiswa? Mengapa harus? mahasiswa lebih banyak menyontek saat ujian online selama masa lockdown pandemi COVID-19? Apa faktor lain yang berkontribusi? untuk menyontek ujian online selama COVID-19 karantina?

Temuan studi penelitian ini dapat membantu universitas dan administrator perguruan tinggi untuk mengerti mengapa menyontek pada ujian online adalah meningkat dan mengapa penting untuk menggunakan penguncian perangkat lunak dan kamera selama ujian online. Ini temuan juga dapat membantu anggota fakultas untuk memeriksa praktik penilaian mereka, untuk mengurangi kecemasan dan stres siswa selama COVID-19 karantina. Dengan cara ini, anggota fakultas bisa mengatasi stres, kecemasan, dan ketidakjujuran akademik di antara siswa, dan mereka juga harus mempertimbangkan jumlah tugas, ujian, dan proyek individu yang siswa diminta untuk lengkap di masa pandemi COVID-19. Selanjutnya, universitas dan perguruan tinggi harus mempertimbangkan integritas akademik dalam silabus mereka, untuk menciptakan budaya yang dapat diterima secara akademis perilaku etis. Dalam hal akademis, studi temuan dapat membantu peneliti lebih memahami konsekuensi COVID-19 di universitas siswa di luar stres, kecemasan, dan depresi. Ini juga menunjukkan bahwa COVID-19 tidak mempengaruhi hanya tingkat kecemasan dan stres di antara siswa tetapi juga perilaku tidak etis mereka. Akhirnya, kerangka kerja yang baru dikembangkan dapat membuka pintu untuk lebih banyak penelitian kuantitatif dan banyak lagi studi yang kuat di seluruh dunia.

Ketidakjujuran akademik mengacu pada perilaku yang ditujukan untuk memberi atau menerima informasi dari orang lain, menggunakan materi yang tidak sah, dan menghindari proses penilaian yang disetujui dalam konteks akademik \citep{faucher2009academic}. Frekuensi ketidakjujuran akademik yang dilaporkan dalam penelitian menunjukkan sifat global dari fenomena ini. Misalnya, dalam sebuah studi oleh Ternes, Babin, Woodworth, dan Stephens \citep{ternes2019academic} 57,3\% siswa pasca sekolah menengah di Kanada mengizinkan siswa lain untuk menyalin pekerjaan mereka. Demikian pula, 61\% mahasiswa sarjana di Swedia menyalin materi untuk tugas kuliah dari buku atau publikasi lain tanpa menyebutkan sumbernya \citep{trost2009psst}. Bekerja bersama dalam sebuah tugas ketika harus diselesaikan sebagai individu dilaporkan oleh 53\% siswa dari empat universitas Australia yang berbeda \citep{brimble2005perceptions}, dan menyalin dari kertas ujian seseorang setidaknya sekali dilakukan oleh 36\% siswa dari empat universitas Jerman \citep{patrzek2015investigating}. Penelitian menunjukkan bahwa ketidakjujuran akademik juga merupakan masalah utama di universitas-universitas Polandia. Dalam studi oleh Lupton, Chapman, dan Weiss \citep{lupton2000international} 59\% siswa mengaku menyontek di kelas saat ini, dan 83,7\% untuk menyontek di beberapa titik selama kuliah. Menurut laporan plagiarisme di Polandia, yang disiapkan oleh Konsorsium Proyek IPPHEAE, 31\% siswa melaporkan menjiplak secara tidak sengaja atau sengaja selama studi mereka \citep{glendinning2015plagiarism}. 

Sistem pencegahan ketidakjujuran akademik yang ada termasuk menggunakan hukuman dan pengawasan \citep{davis1992academic}, menginformasikan siswa tentang perbedaan antara tindakan akademik yang jujur dan tidak jujur \citep{belter2009strategy}, mengadopsi kode kehormatan universitas \citep{mccabe2004ten}, dan mendidik siswa tentang cara menulis makalah dan melakukan penelitian dengan benar \citep{owens20135}. Meskipun metode ini mengarah pada pengurangan ketidakjujuran akademik (lihat \citep{cronan2017changing}), aspek bermasalah mereka termasuk kemungkinan mencapai hanya perubahan sementara dalam perilaku, dampak terbatas pada sikap siswa terhadap menyontek, dan periode implementasi yang lama \citep{crown1998learning} \citep{roig2006attitudes}. Kemungkinan alasan untuk kesulitan ini termasuk fakta bahwa metode pencegahan konvensional jarang mengatasi perbedaan kepribadian dan motivasi akademik siswa, yang mungkin terkait dengan kecenderungan untuk menyontek. Misalnya, penelitian sebelumnya telah melaporkan bahwa emosi negatif dikaitkan dengan sikap positif terhadap plagiarisme \citep{tindall2020negative}; motivasi intrinsik dikaitkan dengan kecurangan yang dilaporkan sendiri lebih rendah \citep{rettinger2004evaluating}; dan nilai-nilai kemanusiaan yang berorientasi sosial negatif, sedangkan nilai-nilai yang terfokus secara pribadi berkorelasi positif dengan ketidakjujuran akademik \citep{koscielniak2019role}.

Penting juga untuk diingat bahwa penerapan metode pencegahan tersebut di atas tidak akan mengurangi ketidakjujuran akademik jika anggota fakultas tidak mengikuti dan menerapkan aturan yang ditetapkan \citep{mccabe1997individual}. Anggota fakultas sering memilih untuk tidak mengambil tindakan formal terhadap siswa yang tidak jujur \citep{chirikov2020role}, dan dalam banyak kasus tidak menggunakan metode yang tersedia bagi mereka untuk mendeteksi dan mencegah kecurangan \citep{sattler2017use}. Namun, ketika mereka menanggapi ketidakjujuran akademik, seringkali dengan cara yang tidak konsisten \citep{mahmud2019students}. Ini mungkin menunjukkan bahwa, ketika berhadapan dengan ketidakjujuran siswa, anggota fakultas lebih memilih untuk memilih metode hukuman dan pencegahan mereka sendiri, yang mungkin berbeda tergantung pada siswa dan profesor tertentu. Jika demikian halnya, maka memeriksa peran perbedaan individu dalam ketidakjujuran akademik dapat berguna tidak hanya untuk lebih memahami sifat pelanggaran akademik tetapi juga untuk mengatasi cara informal fakultas dalam menangani kecurangan siswa.

Dalam konseptualisasi triarkis psikopati, keberanian mewakili keyakinan diri, keberanian, dan toleransi yang tinggi terhadap stres dan ketidakbiasaan; kekejaman menangkap kekurangan interpersonal seperti kurangnya empati, sikap tidak berperasaan dan eksploitatif; dan disinhibisi mewakili kecenderungan impulsif, pengaturan diri yang buruk dan fokus pada kepuasan langsung. Karena mekanisme neurobiologis yang berbeda yang mengarah pada pembentukan aspek-aspek tersebut \citep{patrick2012conceptualizing}, tampaknya kecenderungan terhadap ketidakjujuran akademik mungkin memiliki etiologi yang berbeda. tergantung pada level mereka. Untuk siswa dengan disinhibisi tinggi, menyontek dapat terjadi karena pengendalian diri yang rendah; bagi mereka yang memiliki kekejaman tinggi dari pemberontakan dengan kecenderungan untuk menggunakan orang lain; dan untuk yang berani dari ketahanan emosional dan pencarian sensasi \citep{curtis2018self} \citep{drislane2014clarifying} \citep{nathanson2006predictors}.

Namun, karena keberanian merupakan keberanian tanpa sosialisasi yang gagal \citep{hall2009interview}, melanggar aturan akademik mungkin bukan cara yang disukai untuk mencari kegembiraan di antara siswa yang berani. Jadi, tujuan pertama kami adalah untuk menguji kekuatan prediksi dari keberanian, kekejaman, dan disinhibisi dalam ketidakjujuran akademis.

Selanjutnya, kami tertarik jika hubungan antara aspek psikopati dan ketidakjujuran akademik akan dimediasi oleh perbedaan individu dalam motivasi untuk penguasaan dan kinerja. Motivasi penguasaan didorong oleh kebutuhan untuk berprestasi dan terkait dengan pembelajaran untuk memperoleh pengetahuan, sedangkan motivasi kinerja diarahkan untuk mengurangi kecemasan dan terkait dengan pembelajaran untuk membuktikan diri kepada orang lain \citep{elliot2008measurement}. Kami mengharapkan mediasi karena beberapa alasan. Pertama, melakukan tindakan yang dimotivasi oleh pencapaian tujuan diprediksi oleh tingkat emosi positif dan negatif dan juga oleh aktivitas sistem aktivasi dan penghambatan perilaku \citep{elliot2002approach}, yang juga berkorelasi dengan dimensi model triarki psikopati \citep{sellbom2013examination}. Kedua, motivasi berprestasi yang tidak terkendali sebagian memediasi hubungan antara psikopati dan ketidakjujuran akademik, menunjukkan peran pencapaian dalam memahami hubungan antara psikopati dan perbedaan individu dalam kecenderungan untuk menyontek \citep{williams2010identifying}. Ketiga, kekejaman dan rasa malu berkorelasi negatif dan keberanian berkorelasi positif dengan kesadaran dan segi-seginya \citep{poy2014ffm} \citep{pilch2015polska}. Fakta ini mungkin memainkan peran penting dalam kemauan siswa untuk mengerahkan dan mengendalikan diri untuk mencapai tujuan akademik dan cara tertentu untuk melakukannya \citep{mccabe2013big}. Selain itu, penelitian tentang orientasi penguasaan-tujuan menunjukkan hal itu berkorelasi negatif dengan ketidakjujuran akademik dan pandangan tentang penerimaan ketidakjujuran akademik \citep{bong2014perfectionism} \citep{van2011win} \citep{yang2013investigation} dan bahwa perubahan dari penguasaan ke lingkungan pembelajaran berbasis kinerja menyebabkan peningkatan tingkat ketidakjujuran \citep{anderman2004changes}.

Pembahasan pada paper ini dimulai dengan analisis mengenai integritas akademik (Bagian \ref{sec:integritasakademik}).
Kemudian dilanjutkan dengan penjelasan mengenai ketidak akademik (Bagian \ref{sec:ketidakjujuranakademik}).
Setelah itu dilanjutkan dengan pembahasan dari cara penanggulangan kecurangan akademik (Bagian \ref{sec:antiketidakjujuranakademik}).
Terakhir, didapatkan kesimpulan dari penelitian yang telah dilakukan (Bagian \ref{sec:kesimpulan}).
